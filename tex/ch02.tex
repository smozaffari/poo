% Chapter 02
\chapter{Parent of Origin Effects on Quantitative Phenotypes in a Founder Population}\label{ch:pogwas}
\section[Abstract]{Abstract\footnotemark}


The impact of the parental origin of associated alleles in GWAS has been largely ignored. Yet sequence variants could affect traits differently depending on whether they are inherited from the mother or the father. To explore this possibility, we studied 21 quantitative phenotypes in a large Hutterite pedigree. We first identified variants with significant single parent (maternal-only or paternal-only) effects, and then used a novel statistical model to identify variants with opposite parental effects. Overall, we identified parent of origin effects (POEs) on 11 phenotypes, most of which are risk factors for cardiovascular disease. Many of the loci with POEs have features of imprinted regions and many of the variants with POE are associated with the expression of nearby genes. Overall, our results indicate that POEs, which can be opposite in direction, are relatively common in humans, have potentially important clinical effects, and will be missed in traditional GWAS. 


\footnotetext{Citation for chapter: }

\section{Introduction}\label{ch02-introduction}
Genome-wide association studies (GWAS) typically treat alleles inherited from the mother and the father as equivalent, although variants can affect traits differently depending on whether they are maternal or paternal in origin. In particular, parent of origin effects (POEs) can result from imprinting, where epigenetic modifications allows for differential gene expression on homologous chromosomes that is determined by the parental origin of the chromosome. Mutations in imprinted genes or regions can result in diseases. For example, two very different diseases, Prader-Willi Syndrome (PWS) and Angelman Syndrome (AS), are due to loss of function alleles in genes within an imprinted region on chromosome 15q11-13. Inheriting a loss of function mutation for the SNRPN gene from the father results in PWS but inheriting a loss of function mutation for the UBE3A gene from the mother results in AS\citep{Peters2014,Falls1999} Long noncoding RNA genes at this and other imprinted regions act to silence (i.e. imprint) genes in cis. Imprinted genes are often part of imprinted gene networks, suggesting regulatory links between these genes \cite{Patten:2016cb,Gabory:2009be,Varrault:2006kn}. More than 200 imprinted loci have been described in humans \cite{Benonisdottir:2016dz} but there are likely many other, as yet undiscovered, imprinted loci. 

Previous studies have utilized pedigrees to test maternal and paternal alleles separately for association with phenotypes or with gene expression to uncover new imprinted loci \citep{Kong:2009kk,Baran:2015cx,Garg2012a,Paper2014b,Benonisdottir:2016dz}. Kong \emph{et al} \citep{Kong:2009kk} discovered one locus associated with breast cancer risk only when the allele is inherited from the father and another locus associated with type 2 diabetes risk only when the allele is inherited from the mother. Garg et al. reported parent-of-origin cis-eQTLs with known or putative novel imprinted genes affecting gene expression\citep{Garg2012a}. Two additional studies by Zoledziewsk et al. and Benonisdottir et al. identified opposite POEs on adult height at known imprinted loci\citep{Zoledziewska:2015do,Benonisdottir:2016dz}. Both studies reported associations with variants at the KCNQ1 gene, and one showed additional opposite POEs with height at two known imprinted loci (IGF2-H19 and DLK1-MEG3)\citep{Benonisdottir:2016dz}These studies provide proof-of-principle that alleles at imprinted loci can show POEs, some with opposite effects, with common phenotypes. 

Many existing studies and methods identify parent of origin effects use case/parent trios or case/mother duos\citep{Chuang:2017kp,Howey:2012hj,Ainsworth:2010bp,Weinberg:1999km,Weinberg:1998cf}. Similar to Kong \emph{et al.} \citep{Kong:2009kk}, our method does not require data on the parent and only uses the parent of origin informative alleles which were assigned and phased using PRIMAL\citep{Livne2015}.  In contrast to Kong \emph{et al.} \citep{Kong:2009kk} which used binary traits, our method tests for parent of origin effects on quantitative traits, similar to Benonisdottir \emph{et al.} \citep{Benonisdottir:2016dz} which tested for parent of origin effects on height.

No previous study has included a broad range of human quantitative phenotypes or has studied genome-wide variants with effects in different directions depending on the parent of origin. To address this possibility, we developed a statistical model that directly compares the effects of the maternal and paternal alleles to identify effects that are different, including those that are opposite. We applied this model in a study of 21 common quantitative traits that were measured in the Hutterites, a founder population of European descent for which we have phased genotype data \citep{Livne2015} We 

\section{Results}\label{ch02-results}

