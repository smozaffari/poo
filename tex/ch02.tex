% Chapter 02
\chapter{Parent of Origin Effects on Quantitative Phenotypes in a Founder Population}\label{ch:pogwas}
\section[Abstract]{Abstract\footnotemark}


The impact of the parental origin of associated alleles in GWAS has been largely ignored. Yet sequence variants could affect traits differently depending on whether they are inherited from the mother or the father. To explore this possibility, we studied 21 quantitative phenotypes in a large Hutterite pedigree. We first identified variants with significant single parent (maternal-only or paternal-only) effects, and then used a novel statistical model to identify variants with opposite parental effects. Overall, we identified parent of origin effects (POEs) on 11 phenotypes, most of which are risk factors for cardiovascular disease. Many of the loci with POEs have features of imprinted regions and many of the variants with POE are associated with the expression of nearby genes. Overall, our results indicate that POEs, which can be opposite in direction, are relatively common in humans, have potentially important clinical effects, and will be missed in traditional GWAS. 


\footnotetext{Citation for chapter: John D Blischak, Ludovic Tailleux,
  Amy Mitrano, Luis B Barreiro, and Yoav Gilad. Mycobacterial
  infection induces a specific human innate immune
  response. Scientific reports, 5:16882, 2015}

\section{Introduction}\label{ch02-introduction}
Genome-wide association studies (GWAS) typically treat alleles inherited from the mother and the father as equivalent, although variants can affect traits differently depending on whether they are maternal or paternal in origin. In particular, parent of origin effects (POEs) can result from imprinting, where epigenetic modifications allows for differential gene expression on homologous chromosomes that is determined by the parental origin of the chromosome. Mutations in imprinted genes or regions can result in diseases. For example, two very different diseases, Prader-Willi Syndrome (PWS) and Angelman Syndrome (AS), are due to loss of function alleles in genes within an imprinted region on chromosome 15q11-13. Inheriting a loss of function mutation for the SNRPN gene from the father results in PWS but inheriting a loss of function mutation for the UBE3A gene from the mother results in AS
\section{Results}\label{ch02-results}

