\abstract

Variants can affect traits differently depending on whether they are inherited from the mother or the father, but genome wide association studies (GWAS) treat maternal and paternal alleles as equivalent. In addition, the variants identified by GWAS do not account for a significant portion of the heritability for the corresponding trait and could be due to underlying biological mechanisms that are not yet well understood. My thesis addresses these limitations by disentangling the effects of maternal and paternal alleles on gene expression as well as on disease associated phenotypes in the Hutterites, a founder population of European descent. With phased genotype data we can ask questions about parent of origin effects in this population. First, I tested for maternal and paternal genetic associations on cardiovascular disease and asthma associated traits and developed a novel method to detect variants that have opposite effects on the trait of interest depending on the parent of origin of the variant. We identified variants that have maternal-only or paternal-only effects, as well as variants that have opposite effects on traits ? which would not be detected in a standard GWAS. this is the largest family based study of parent of origin effects on quantitative traits and the first to look for opposite parental effects. In the second chapter, I map RNA-seq from lymphoblastoid cell lines (LCLs) to parental haplotypes in 306  Hutterites and detect known imprinted genes and two novel imprinted genes (\emph{PXDC1} and \emph{PWAR6}) in known imprinted regions. These imprinting gene patterns are validated using parent of origin expression from peripheral blood leukocyte (PBL) RNA-seq from 99 different Hutterites; imprinting control regions near the novel genes are validated using PBL methylation in the same 99 Hutterites. Finally, I explore searching for parent of origin effects on gene expression, or parent of origin eQTLs, first for opposite effects and then maternal and paternal specific effects. 