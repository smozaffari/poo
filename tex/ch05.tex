% !TEX encoding = UTF-8 Unicode
\chapter{Conclusion}

As we look past the "low hanging fruit" of GWAS, we turn to other biological mechanisms where genetics can influence our traits, including studies of rare variation\citep{Igartua:2017ir,Li:2017bq}, gene by environment interactions, and parent of origin effects, among others. The Hutterites are an ideal population to look for parent of origin effects on quantitative disease related traits and gene expression because of their common environmental exposures and similar genetic background\cite{Weiss:2005cq,Abney2001,Ober:2001dy}. With the availability of RNA-seq data, novel methods for imputing and phasing data at a population scale\citep {Livne2015}, and extensive phenotyping of the Hutterites, we are able to start investigating such questions related to parent of origin effects. 



\section{A novel method to detect opposite effects of parentally inherited variants on cardiovascular disease and asthma associated traits}
 
 In Chapter \ref{ch:pogwas}, I describe our study to detect genetic variation than can impact cardiovascular disease and asthma associated traits when inherited from one parent and not the other. In order to maximize our data, we developed a novel method that can be used to detect parentally inherited alleles with opposite effects on quantitative traits, depending on which parent they were inherited from. These associations would be missed in traditional GWAS. Previous studies have explored parent of origin effects but mostly using trios \cite{Garg2012a,Ainsworth:2010bp,Howey:2012hj} and on binary disease status \cite{Kong:2009kk,Ainsworth:2010bp}. Only a few studies have searched for parent of origin effects on quantitative traits, starting with height \cite{Benonisdottir:2016dz,Zoledziewska:2015do}.
 
Using our method, we identify parent of origin effects (POEs) on 11 phenotypes in the Hutterites, most of which are risk factors for cardiovascular disease. Most of the loci we identified have features of imprinted regions and many of the variants are associated with expression of nearby genes. Most of the phenotypes are also associated with metabolic traits consistent with the parent conflict hypothesis of imprinted genes first put forward by Haig \citep{Haig:2000if,Barlow:2014dv,Patten:2016cb}. The idea suggests that parental interests may be in conflict such that paternal alleles favor growth of the fetus at the expense of the mother while maternal alleles favor restricting resources to the fetus to ensure preservation of her own nutritional needs. We show that POEs, which can be opposite in direction, are relatively common in humans, are possible imprinted regions, and have potentially important clinical effects. 
 
It is necessary to replicate the parent of origin effects we identified in a different population to verify these effects exist. The associations did not replicate in the Sardinia population, although there were few suggestively significant. Additionally a power analysis will need to be done to confirm our method would pick up genetic variants of the effect size and minor allele frequency that we identified.

It is important to further investigate these regions. One of the regions with a POE identified in our study with \emph{LINC01081} has been studied in detail by another group showing that a parent of origin effect exists at this region \cite{Szafranski:2016fz}. Others have shown POEs at known imprinted regions that affect height\cite{Benonisdottir:2016dz,Zoledziewska:2015do} that we were not able to replicate in the Hutterites likely due to our smaller sample size. It would be the next step to show whether or not the regions we identify with POEs lie in or near previously unidentified imprinted regions.
 
\section{Identifying two novel imprinted genes in known imprinted regions using parent of origin gene expression}

In Chapter \ref{ch:imprinted}, I identified two new imprinted genes using parent of origin and allele specific expression. Using a novel method as a variation on WASP \cite{vandeGeijn:2015hi}, I mapped RNA-seq reads to parental haplotypes using SNPs in the reads and parent of origin information of SNPs. We identified known imprinted genes and two novel imprinted genes, \emph{PXDC1} and \emph{PWAR6}, among our genes with asymmetrical parent of origin gene expression. We validated the patterns of gene expression using RNA-seq from peripheral blood leukocytes (PBLs). To validate the imprinted genes further, we used DNA methylation levels in the PBLs to confirm imprinting control regions (ICRs) previously defined\cite{Joshi:2016bb,Court:2014kc}. Our two new imprinted genes lie in known imprinted regions with known ICRs providing more evidence for their imprinting status.
 
This is the largest pedigree based genome-wide scan for imprinted regions to date with 306 Hutterite individuals. We also provide a new way of mapping reads to parent of origin haplotypes and identify two new possible imprinted genes.

We would still need to further validate \emph{PXDC1} and \emph{PWAR6} as imprinted genes in a different population and possibly in additional cell types. Further characterization of these loci are still required. We identified these genes as imprinted but had overall very few parentally mapped reads. This would need to be replicated with many more parentally mapped reads. It was somewhat surprising they were not previously discovered as imprinted genes since they are in known and somewhat well characterized imprinted regions. It is possible they are tissue and/or developmentally specific as has been shown for other imprinted genes \cite{Baran:2015cx}.  It is also possible that, although imprinted, variation at these imprinted genes does not affect disease, contrary to other imprinted genes and how they have been previously identified. 

\section{Can genetic variation by parent of origin influence gene expression on the same haplotype?}

In Chapter \ref{ch:poeqtl}, I explored if parentally inherited genetic variation can affect gene expression as well as haplotype specific gene expression (maternal and paternal specific expression). Using the method described in Chapter \ref{ch:pogwas} to detect opposite effect associations, we did not identify any SNPs that had opposite effects on total gene expression depending on which parent the allele as inherited from. To get at the same opposite effects in a different way, we performed a maternal eQTL (mat-eQTL) with maternal alleles on maternal expression as well as a paternal eQTL (pat-eQTL) with paternal alleles on paternal expression. Across all the SNP-gene pairs analyzed we did not identify any effects that were opposite in direction. 

To determine if any alleles had a specific effect on gene expression when inherited from one parent or the other, we used a parent of origin allele specific expression (PO-ASE) test and identified alleles that when inherited from one parent, and not the other, had a different effect on gene expression from the same haplotype as the allele. 

While our model detected genes that look like they could have a PO allele specific effect, there is still room for improvement. We use gene expression counts but our model is not accounting for overdispersion, which needs to be accounted for when looking at gene expression. We also looked for PO-ASE effects with very few parentally mapped reads. For standard eQTL studies we require at least 10 million reads, whereas, here, we have on average 1.8\% of 10 million reads mapped to a parental haplotype resulting in, on average, about half of 1.8 million reads to use for an eQTL study. Our study is likely, therefore, underpowered to identify SNPs that have opposite effects on gene expression depending on which parent it was inherited from. More reads are required to perform this study and find parent specific effects. 
 
Even with these parent of origin eQTLs identified in Chapter \ref{ch:poeqtl}, it is not clear what biological mechanism could lead to such an effect. It is possible that the parentally inherited allele that contributes to gene expression differently from the rest could disrupt normal expression by inducing silencing of the gene in a chromosome specific manner. It is also possible that the opposite allele - from the same parent - results in abberant over expression of the allele. Both of these would result in a parent of origin effect on gene expression but why or how these would occur is unknown. 

\section{Future Directions}

With this large pedigree data we are able to answer some questions about parent of origin effects on gene expression and quantitative traits, but it barely scratches the surface of biological mechanisms that we don't yet have a pipeline for or the ideal dataset to detect such effects, in contrast to genome wide association studies. From Chapter \ref{ch:pogwas}, our method to detect opposite parent of origin effects has been progress in this direction. The method needs to be tested on multiple datasets to confirm the results are accurate. There would also need to be a power analysis done to confirm that most of the associations are not false positives. Although these opposite effect variants don't show an effect on gene expression in our study or others \cite{Benonisdottir:2016dz}, we still need to understand the mechanism by which these variants are acting. Better characterization of the expression of genes nearby these variants needs to be done, across different tissues, to get at whether these effects are a result of imprinting. Methylation and chromatin in these regions could be studied once the gene or genes contributing to the parent of origin phenotype are identified. Characterization of allele specific interactions using chromatin conformation capture at heterozygous SNPs could also help unravel the different interactions among parental alleles if one is not imprinted and both genes are actively expressed. Ultimately, functional characterization of these SNPs remains necessary to understand the mechanism behind these opposite parental effects and the traits they impact.

In Chapter \ref{ch:imprinted} we identify known imprinted genes in lymphoblastoid cell lines as well as two novel imprinted genes. Using a larger dataset, with accurate parent of origin calls, more parent of origin calls, more sequencing reads, and across different tissues is necessary to confirm these novel imprinted genes as well as confirm more known imprinted genes. Imprinted genes have been shown to be tissue specific \cite{Baran:2015cx} and thus performing a similar study in more tissues and samples could increase the resolution on imprinted genes and their mechanisms of gene expression. 

Although we were not able to do it in Chapter \ref{ch:poeqtl}, it remains necessary to identify variation that can impact imprinted genes or impact gene expression by parent of origin. Similar to diseases that result from mutations in imprinted genes, it is possible genetic variation that affects gene expression in a parent of origin manner can affect diseases or disease associated traits. 

In this dissertation, I characterize parent of origin effects of some variants, and their small effects on quantitative disease associated traits, but there still remains a lot to be discovered and better understood for us to have a better understanding of the genome and it's impact on human health and disease. These parent of origin effects, although small, likely do contribute to heritability of traits as more researchers, including us, have identified genetic variation that can impact traits by parent of origin \cite{Benonisdottir:2016dz,Zoledziewska:2015do,Garg2012a,Kong:2009kk,Mozaffari:dg}. 


\section{Concluding remarks}

In this thesis, I contribute to the growing knowledge of additional biological mechanisms, specifically, parent of origin effects, that can contribute to phenotypes and disease risk missed in standard GWAS. In all the chapters of this dissertation we are able to leverage the parent of origin information for alleles in the Hutterites and use them to uncover novel effects of genetic variation on gene expression and quantitative traits. In Chapter \ref{ch:pogwas} I developed a new method to test for opposite effects of parentally inherited genetic variation on quantitative traits and we find 11 parent of origin effects, including maternal, paternal, and opposite parental effects. In Chapter \ref{ch:imprinted}, I mapped RNA-seq to parental haplotypes using parent of origin of alleles and uncover known and new imprinted genes (\emph{PXDC1} and \emph{PWAR6}), which show similar patterns of expression in PBLs and have well characterized ICRs in the region. In Chapter \ref{ch:poeqtl}, I explored how parent of origin genetic variation could influence gene expression. We did not find any opposite effects on gene expression but identified alleles that when inherited from one parent showed altered gene expression patterns. 

Throughout this research, I investigated the basis for parent of origin effects for which we have not had the methodology and or the appropriate dataset I have today to explore such a phenomenon. The novelty of these data and the questions I asked  has allowed for the expansion of this research from one original aim to a full dissertation covering 3 separate projects and future directions such as measuring parent of origin heritability on traits and gene expression. The research in this dissertation provides methods and a basis for studying parent of origin effects and other less well characterized effects that could contribute to heritability of traits and disease risk. 




