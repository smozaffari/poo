% !TEX encoding = UTF-8 Unicode
\chapter{Conclusion}

As we look past the "low hanging fruit" of GWAS, we turn to other biological mechanisms where genetics can influence our traits, including studies or rare variation\cite{Igartua:2017ir}, gene by environment interactions, and parent of origin effects among others. The Hutterites that the Ober Lab studies are an ideal population to look for parent of origin effects on quantitative disease related traits and gene expression\cite{Weiss:2005cq,Abney2001,Ober:2001dy}. With RNA-seq and novel methods for imputing and phasing data at a population scale\citep{Livne2015}, we are able to start getting at these questions. 


\section{A novel method to detect opposite effects of parentally inherited variants on cardiovascular disease and asthma associated traits}
 
 In Chapter \ref{ch:pogwas}, I describe our study to detect genetic variation than can impact cardiovascular disease and asthma associated traits when inherited from one parent and not the other. In order to maximize on our data, we develop a novel method in this chapter that can be used to detect if parentally inherited alleles can have opposite effects on quantitative traits depending on which parent they were inherited from. These associations would be missed in traditional GWAS. Previous studies have explored parent of origin effects but mostly using trios\cite{Garg2012a,Ainsworth:2010bp,Howey:2012hj} and on case control status\cite{Kong:2009kk,Ainsworth:2010bp}. Only a few have begun to search for parent of origin effects on quantitative traits\cite{Benonisdottir:2016dz,Zoledziewska:2015do}.
 
 We identify parent of origin effects (POEs) on 11 phenotypes in the Hutterites, most of which are risk factors for cardiovascular disease. Most of the loci we identified have features of imprinted regions and many of the variants are associated with expression of nearby genes. Overall, we show that POEs which can be opposite in direction, are relatively common in humans and have potentially important clinical effects. 
 
 \subsection{Future Directions}
It is necessary to still replicate the results in a different population to verify these effects exist. They did not replicate in the Sardinia population. Additionally a power analysis should be done to confirm our method would pick up genetic variants of the effect size and minor allele frequency that we identified.

It is important to further investigate these regions. One of the regions with a POE identified in our study, \emph{LINC01081} has been studied in detail by another group showing that a parent of origin effect exists at this region independent of our study\cite{Szafranski:2016fz}. Others have shown POEs at known imprinted regions that affect height\cite{Benonisdottir:2016dz,Zoledziewska:2015do}. It would be the next step to show that the regions we identify with POEs could be at imprinted regions.
 
\section{Identifying two novel imprinted genes in known imprinted regions using parent of origin gene expression}

 In Chapter \ref{ch:imprinted}, I identified two new imprinted genes using parent of origin and allele specific expression. Using a novel method as a variation on WASP\cite{vandeGeijn:2015hi}, I mapped RNA-seq reads to parental haplotypes using SNPs in the reads and parent of origin information of SNPs. We identified known and novel imprinted genes among our genes with asymmetrical parent of origin gene expression. We validated the patterns of gene expression using RNA-seq from peripheral blood leukocytes (PBLs). To validate the imprinted genes further, we used DNA methylation levels in the PBLs to confirm imprinting control regions (ICRs) previously defined\cite{Joshi:2016bb,Court:2014kc}.
 
This is the largest pedigree based genome-wide scan for imprinted regions with 306 Hutterite individuals. 

 \subsection{Future Directions}




\section{Identifying two novel imprinted genes in known imprinted regions using parent of origin gene expression}


 \subsection{Future Directions}

\section{Concluding remarks}

